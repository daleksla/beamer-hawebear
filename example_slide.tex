\documentclass[11pt,t,aspectratio=169]{beamer} % set document class from standard to beamer document
\setbeamersize{text margin left=1em,text margin right=1em}
\usetheme{hawebear} % import theme

\usepackage{hyperref}
\usepackage{physics}
\usepackage{multicol}
\usepackage{minted}
%\usefonttheme{professionalfonts}
\usefonttheme{serif}

% This is used as a banner to denote a section
% Delete this, if you do not want the table of contents to pop up at the beginning of each sectionn
\AtBeginSection[]
{
    \begingroup
        \setbeamertemplate{background canvas}[vertical shading][bottom=black,top=black]
        \setbeamertemplate{footline}[sectionfootline] 
        \setbeamertemplate{section page}[mysection]
        \frame[c]{
            \sectionpage
        }
    \endgroup
}

\title{Project Proposal Idea}
\subtitle{A Comparison of High-Level Parallel Programming Frameworks}
\author{Salih Ahmed}
\institute{University of Birmingham}
\date{\today}

%titlepage logo
\titlegraphic{ % you can include more then one logo if you want
    \includegraphics[scale=0.15]{logo.png}
    \includegraphics[scale=0.15]{logo.png}
}

\begin{document}

\begin{frame} % create titlepage
    \titlepage
\end{frame}


\begin{frame}[c]{Outline} % create first slide
    \tableofcontents % dump contents (ie sections)
\end{frame}

\section{What is this?} % start section, which implicitly inserts section slide between contentful slides
                     % just use subsection if you don't want title slides
                     % also use subsections if you want the slide to appear on the contents page under the section
                     % also use subsections to logically contain multiple frames / slides under the same subtopic
    \begin{frame}[c]
        \frametitle{Disclaimer} % title of one, contentful slide
        \framesubtitle{Generic subtitle} % subtitle of one, contentful slide
        \begin{multicols}{2}
        \begin{itemize}
            \item THIS IS NOT AN OFFICIAL LEHIGH TEMPLATE.
            \item Modified from the original \LaTeX{} template by Alex Pacheco.
            \item Followed the latest \href{https://www.lehigh.edu/~inis/sdownloads/Lehigh-Brand-Visual-Guide.pdf}{Lehigh University Branding \& Visual Identity Guide}.
            \item Modify if you have suggestions to improve it and pass on the modifications so that anyone interested in using \LaTeX{} for presentations could benefit.
            \end{itemize}
        \end{multicols}
        \begin{itemize}
            \item Typing equations with \emph{physics} package is always more convenient.
            $$\pdv{\vb*{U}}{t}+(\vb*{U}\cdot\nabla)\vb*{U}-\nu\laplacian\vb*{U} =-\frac{1}{\rho }\grad p+\vb*{g}$$
        \end{itemize}
    \end{frame}

    \begin{frame}[c]
        \frametitle{Todo} % title of one, contentful slide
        \begin{itemize}
            \item Brown -> black
            \item change name from leigh document to something else
        \end{itemize}
    \end{frame}

\section{Using template}


    \begin{frame}[c,fragile]
        \frametitle{How to use it}
        \framesubtitle{Classic LaTex}

        You can use sections to create logical sections (which can be seen in title pages, etc.), itemize / enumerate to create lists, etc..

        Alternative, you can just type a bunch of text as has been exhibited in this slide - more useful for larger blocks of text
    \end{frame}

    \begin{frame}[c,fragile]
        \frametitle{How to use it}
        \framesubtitle{Blocks}
        \begin{block}{This is a block}
            \begin{minted}{latex}
\usepackage{minted}
...
\begin{frame}[c,fragile] % Use fragile option
            \end{minted}
        \end{block}
        
        \begin{exampleblock}{This is an example block}
            \begin{minted}{python}
import numpy as np
a = np.array([1, 2, 3])
            \end{minted}
        \end{exampleblock}
        
        \begin{alertblock}{This is an alert block}
            \begin{minted}{bash}
fatal error: mpi.h: No such file or directory
compilation terminated.
            \end{minted}
        \end{alertblock}
    
    \end{frame}

\end{document}